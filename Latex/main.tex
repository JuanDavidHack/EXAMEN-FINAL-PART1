\documentclass[12pt,a4paper]{article}
\usepackage[utf8]{inputenc}
\usepackage{amsmath, amssymb}
\usepackage{graphicx}
\usepackage{hyperref}
\usepackage{geometry}
\usepackage{enumitem}
\usepackage{float}
\usepackage{booktabs}
\geometry{margin=2.5cm}

% Configuración de listings para MATLAB (para futuros usos)
\usepackage{listings}
\usepackage{xcolor}
\lstset{
  language=Matlab,
  basicstyle=\ttfamily\small,
  keywordstyle=\color{blue}\bfseries,
  commentstyle=\color{green!50!black},
  stringstyle=\color{red},
  numbers=left,
  numberstyle=\tiny,
  frame=single,
  breaklines=true
}

\title{Ejercicios de Variable Compleja en MATLAB}
\author{Juan González \\ 20232025089
\\
Yesid González \\ 20232025073}
\date{\today}

\begin{document}

\maketitle

% Tabla de contenidos
\tableofcontents
\newpage

\section{Introducción}
El presente documento tiene como objetivo recopilar y desarrollar una serie de ejercicios relacionados con la teoría de variable compleja, utilizando como herramienta principal el software \texttt{MATLAB}.  
La variable compleja constituye un campo fundamental dentro del análisis matemático, con aplicaciones en la ingeniería, la física y diversas áreas de la ciencia. En este trabajo se abordan problemas clásicos como operaciones con números complejos, representaciones en el plano, transformaciones conformes, verificación de las ecuaciones de Cauchy-Riemann, así como el cálculo de integrales de contorno y límites.

El uso de \texttt{MATLAB} permite no solo realizar los cálculos simbólicos y numéricos de manera eficiente, sino también visualizar gráficamente mapeos y funciones complejas, lo cual enriquece la comprensión de los conceptos teóricos.

\newpage

% =============================================================================
% EJERCICIOS 1-18
% =============================================================================
\section{Ejercicios 1-18}

\subsection{Ejercicio 1: Operaciones básicas con complejos}
\subsubsection{Enunciado}
Dados \(z_1=3+2i\) y \(z_2=1-5i\), calcular:
\[
a)\; z_1+z_2, \quad b)\; z_1 \cdot z_2, \quad c)\; \frac{z_1}{z_2}.
\]

\subsubsection{Solución matemática}
\begin{align*}
a)\;z_1+z_2 &= 4-3i, \\
b)\;z_1 z_2 &= 13-13i, \\
c)\;\frac{z_1}{z_2} &= \frac{-7+17i}{26} = -\tfrac{7}{26}+\tfrac{17}{26}i.
\end{align*}

% -----------------------------------------------------------------------------
\subsection{Ejercicio 2: Forma polar de complejo}
\subsubsection{Enunciado}
Calcule la forma polar de \(z = -2 + 2i\).

\subsubsection{Solución matemática}
\[
r = |z| = \sqrt{(-2)^2 + (2)^2} = 2\sqrt{2}, \quad
\theta = \frac{3\pi}{4}.
\]
Por tanto,
\[
z = 2\sqrt{2}\,(\cos\tfrac{3\pi}{4} + i\sin\tfrac{3\pi}{4})
= 2\sqrt{2}\,e^{i\frac{3\pi}{4}}.
\]

\subsubsection{Resultado}
\[
r = 2\sqrt{2}, \quad \theta = \tfrac{3\pi}{4} \; (135^\circ).
\]

% -----------------------------------------------------------------------------
\subsection{Ejercicio 3: Raíces quintas de la unidad}
\subsubsection{Enunciado}
Grafique las raíces quintas de la unidad.

\subsubsection{Solución matemática}
Las raíces quintas de la unidad son
\[
z_k = e^{2\pi i k / 5}, \quad k=0,1,2,3,4,
\]
las cuales se distribuyen uniformemente en la circunferencia unitaria del plano complejo.

\subsubsection{Gráfica}
\begin{figure}[H]
  \centering
  \includegraphics[width=0.7\textwidth]{figuras/raices_quintas.png}
  \caption{Representación de las cinco raíces de la unidad sobre la circunferencia unitaria}
\end{figure}

% -----------------------------------------------------------------------------
\subsection{Ejercicio 4: Ecuaciones de Cauchy-Riemann}
\subsubsection{Enunciado}
Verifique las ecuaciones de Cauchy–Riemann para la función \( f(z)=e^{z} \).

\subsubsection{Solución matemática}
Escribiendo \(z=x+iy\):
\[
f(z)=e^{x+iy}=e^{x}(\cos y+i\sin y),
\quad u=e^{x}\cos y,\; v=e^{x}\sin y.
\]
Cálculo directo de derivadas:
\[
u_x=e^{x}\cos y,\; u_y=-e^{x}\sin y,\; v_x=e^{x}\sin y,\; v_y=e^{x}\cos y.
\]
Como \(u_x=v_y\) y \(u_y=-v_x\) para todo \(x,y\), \(f\) cumple Cauchy–Riemann y por tanto es analítica en \(\mathbb{C}\).

% -----------------------------------------------------------------------------
\subsection{Ejercicio 5: Integral de línea}
\subsubsection{Enunciado}
Calcule \(\displaystyle \int_{C} z^{2}\,dz\), donde \(C\) es el segmento desde \(0\) hasta \(1+i\).

\subsubsection{Resultado en MATLAB}
\begin{figure}[H]
  \centering
  \includegraphics[width=0.8\textwidth]{figuras/ej5_matlab.png}
  \caption{Resultado de la integral de línea \(\int_{C} z^{2}\,dz\) mostrado en MATLAB}
\end{figure}

% -----------------------------------------------------------------------------
\subsection{Ejercicio 6: Serie de Taylor}
\subsubsection{Enunciado}
Desarrolle en serie de Taylor alrededor de \(z=0\) la función
\[
f(z)=\frac{1}{1+z}.
\]

\subsubsection{Solución matemática}
Para \(|z|<1\) se tiene la expansión geométrica
\[
\frac{1}{1+z}=\sum_{n=0}^{\infty}(-1)^n z^n = 1 - z + z^2 - z^3 + \cdots.
\]

\subsubsection{Resultado en MATLAB}
\begin{figure}[H]
  \centering
  \includegraphics[width=0.8\textwidth]{figuras/output_N6.png}
  \caption{Desarrollo en serie de Taylor en MATLAB}
\end{figure}

% -----------------------------------------------------------------------------
\subsection{Ejercicio 7: Mapeo \(w=z^2\)}
\subsubsection{Enunciado}
Visualice el mapeo \(w=z^{2}\) para una cuadrícula en el plano \(z\).

\subsubsection{Solución matemática}
La transformación \(w = z^2\) es una función analítica que mapea el plano complejo \(z = x + iy\) al plano \(w = u + iv\) mediante:
\[
u = x^2 - y^2, \quad v = 2xy.
\]
Esta transformación duplica los ángulos en el origen y convierte líneas rectas en parábolas.

\subsubsection{Resultado en MATLAB}
\begin{figure}[H]
  \centering
  \includegraphics[width=0.8\textwidth]{figuras/output_N7.png}
  \caption{Mapeo \(w=z^2\) para una cuadrícula en el plano \(z\)}
\end{figure}

% -----------------------------------------------------------------------------
\subsection{Ejercicio 8: Límite complejo}
\subsubsection{Enunciado}
Calcule \(\lim_{n\to\infty}\left(\frac{n+1}{n}\right)^{ni}\).

\subsubsection{Solución matemática}
\[
\lim_{n\to\infty}\left(\frac{n+1}{n}\right)^{ni} = e^{i} = \cos(1) + i\sin(1)
\]

\subsubsection{Resultado en MATLAB}
\begin{figure}[H]
  \centering
  \includegraphics[width=0.8\textwidth]{figuras/output_N8.png}
  \caption{Cálculo del límite complejo en MATLAB}
\end{figure}

% -----------------------------------------------------------------------------
\subsection{Ejercicio 9: Curvas de nivel}
\subsubsection{Enunciado}
Grafique las curvas de nivel de \(\operatorname{Re}(e^{1/z})\).

\subsubsection{Solución matemática}
\[
\operatorname{Re}(e^{1/z}) = e^{\frac{x}{x^2 + y^2}} \cos\left(\frac{y}{x^2 + y^2}\right)
\]
donde \(z = x + iy\). Las curvas de nivel muestran líneas de valor constante de esta función.

\subsubsection{Resultado en MATLAB}
\begin{figure}[H]
  \centering
  \includegraphics[width=0.7\textwidth]{figuras/output_N9.png}
  \caption{Curvas de nivel de \(\operatorname{Re}(e^{1/z})\)}
\end{figure}

% -----------------------------------------------------------------------------
\subsection{Ejercicio 10: Raíces de ecuación}
\subsubsection{Enunciado}
Resuelva la ecuación \(z^{4}=16i\).

\subsubsection{Solución matemática}
Expresando \(16i\) en forma polar:
\[
16i = 16e^{i\pi/2} = 16e^{i(\pi/2 + 2k\pi)}
\]
Las raíces cuartas son:
\[
z_k = 2e^{i(\pi/8 + k\pi/2)}, \quad k=0,1,2,3
\]

\subsubsection{Resultado en MATLAB}
\begin{figure}[H]
  \centering
  \includegraphics[width=0.8\textwidth]{figuras/output_N10.png}
  \caption{Raíces de \(z^{4}=16i\) en MATLAB}
\end{figure}

% Continúa con los ejercicios 11-18 en el mismo formato...
% Por brevedad, muestro solo la estructura

\subsection{Ejercicio 11: Integral de \(e^{i\theta}\)}
\subsubsection{Enunciado}
Calcule la integral \(\int_{0}^{2\pi}e^{i\theta}d\theta\).

\subsubsection{Solución matemática}
\[
\int_{0}^{2\pi}e^{i\theta}d\theta = \int_{0}^{2\pi}(\cos\theta + i\sin\theta)d\theta = 0
\]
ya que las integrales de \(\cos\theta\) y \(\sin\theta\) sobre un periodo completo son cero.

\subsubsection{Resultado en MATLAB}
\begin{figure}[H]
  \centering
  \includegraphics[width=0.8\textwidth]{figuras/output_N11.png}
  \caption{Integral de \(e^{i\theta}\) en MATLAB}
\end{figure}

\subsection{Ejercicio 12: Integral por residuos}
\subsubsection{Enunciado}
Evalúe \(\int_{C}\frac{\cos z}{z}dz\) donde \(C\) es el círculo \(|z|=1\).

\subsubsection{Solución matemática}
Por el teorema de los residuos, el residuo en \(z=0\) es:
\[
\operatorname{Res}(f,0) = \lim_{z\to 0} z\cdot\frac{\cos z}{z} = \cos(0) = 1
\]
Por tanto:
\[
\int_{C}\frac{\cos z}{z}dz = 2\pi i \cdot \operatorname{Res}(f,0) = 2\pi i
\]

\subsubsection{Resultado en MATLAB}
\begin{figure}[H]
  \centering
  \includegraphics[width=0.8\textwidth]{figuras/output_N12.png}
  \caption{Integral de \(\frac{\cos z}{z}\) por residuos en MATLAB}
\end{figure}

% -----------------------------------------------------------------------------
\subsection{Ejercicio 13: Derivada del logaritmo}
\subsubsection{Enunciado}
Derivada de \(f(z)=\log(z)\).

\subsubsection{Solución matemática}
Para la rama principal del logaritmo complejo:
\[
\frac{d}{dz}\log(z) = \frac{1}{z}
\]
válido para \(z \neq 0\) y evitando el corte de rama.

\subsubsection{Resultado en MATLAB}
\begin{figure}[H]
  \centering
  \includegraphics[width=0.8\textwidth]{figuras/output_N13.png}
  \caption{Derivada de \(\log(z)\) en MATLAB}
\end{figure}

% -----------------------------------------------------------------------------
\subsection{Ejercicio 14: Serie de Taylor de \(e^z\)}
\subsubsection{Enunciado}
Calcule \(\sum_{n=0}^{\infty}\frac{z^{n}}{n!}\) y verifique que es \(e^{z}\).

\subsubsection{Solución matemática}
La serie:
\[
\sum_{n=0}^{\infty}\frac{z^{n}}{n!} = 1 + z + \frac{z^2}{2!} + \frac{z^3}{3!} + \cdots
\]
es la definición de la función exponencial compleja \(e^{z}\), que es analítica en todo el plano complejo.

\subsubsection{Resultado en MATLAB}
\begin{figure}[H]
  \centering
  \includegraphics[width=0.8\textwidth]{figuras/output_N14.png}
  \caption{Serie de Taylor de \(e^z\) en MATLAB}
\end{figure}

% -----------------------------------------------------------------------------
\subsection{Ejercicio 15: Raíces de polinomio}
\subsubsection{Enunciado}
Encuentre y grafique las raíces de \(z^{3}+z^{2}+z+1=0\).

\subsubsection{Solución matemática}
Factorizando:
\[
z^3 + z^2 + z + 1 = (z+1)(z^2 + 1) = (z+1)(z+i)(z-i)
\]
Las raíces son: \(z = -1, i, -i\).

\subsubsection{Resultado en MATLAB}
\begin{figure}[H]
  \centering
  \includegraphics[width=0.8\textwidth]{figuras/output_N15.png}
  \caption{Raíces de \(z^{3}+z^{2}+z+1=0\) en MATLAB}
\end{figure}

% -----------------------------------------------------------------------------
\subsection{Ejercicio 16: Mapeo \(w=\sin(z)\)}
\subsubsection{Enunciado}
Mapeo de \(w=\sin(z)\) para \(z=x+iy\) con \(x\in[-\pi,\pi],y\in[0,1]\).

\subsubsection{Solución matemática}
Usando la identidad:
\[
\sin(z) = \sin(x)\cosh(y) + i\cos(x)\sinh(y)
\]
El mapeo transforma rectángulos en el plano \(z\) en regiones elípticas en el plano \(w\).

\subsubsection{Resultado en MATLAB}
\begin{figure}[H]
  \centering
  \includegraphics[width=0.8\textwidth]{figuras/output_N16.png}
  \caption{Mapeo \(w=\sin(z)\) en MATLAB}
\end{figure}

% -----------------------------------------------------------------------------
\subsection{Ejercicio 17: Integral con residuos}
\subsubsection{Enunciado}
Calcule \(\int_{C}\frac{dz}{z^{2}+4}\) donde \(C\) es el círculo \(|z-i|=2\).

\subsubsection{Solución matemática}
Los polos están en \(z = \pm 2i\). Solo \(z = 2i\) está dentro del contorno.
\[
\operatorname{Res}(f,2i) = \lim_{z\to 2i} (z-2i)\cdot\frac{1}{(z-2i)(z+2i)} = \frac{1}{4i}
\]
\[
\int_{C}\frac{dz}{z^{2}+4} = 2\pi i \cdot \frac{1}{4i} = \frac{\pi}{2}
\]

\subsubsection{Resultado en MATLAB}
\begin{figure}[H]
  \centering
  \includegraphics[width=0.8\textwidth]{figuras/output_N17.png}
  \caption{Integral de \(\frac{1}{z^{2}+4}\) en MATLAB}
\end{figure}

% -----------------------------------------------------------------------------
\subsection{Ejercicio 18: Gráfica de \(f(z)=z^2\)}
\subsubsection{Enunciado}
Grafique la función \(f(z)=z^{2}\) en el dominio \([-2,2]\times[-2,2]\).

\subsubsection{Solución matemática}
La función \(f(z)=z^2\) mapea el plano complejo según:
\[
(x + iy)^2 = (x^2 - y^2) + i(2xy)
\]
Se pueden visualizar las partes real e imaginaria, o el módulo y argumento.

\subsubsection{Resultado en MATLAB}
\begin{figure}[H]
  \centering
  \includegraphics[width=0.8\textwidth]{figuras/output_N18.png}
  \caption{Gráfica de \(f(z)=z^2\) en MATLAB}
\end{figure}

\newpage

% =============================================================================
% EJERCICIOS 19-37
% =============================================================================
\section{Ejercicios 19-37}

\subsection{Ejercicio 19: Derivada de función exponencial}
\subsubsection{Enunciado}
Calcule la derivada de \(f(z)=3ie^{z}\).

\subsubsection{Solución matemática}
\[
f'(z) = \frac{d}{dz}(3ie^{z}) = 3i \cdot e^{z}
\]

\subsubsection{Resultado}
\[
f'(z) = 3ie^{z}
\]

\subsubsection{Resultado en MATLAB}
\begin{figure}[H]
  \centering
  \includegraphics[width=0.8\textwidth]{figuras/output_N19.png}
  \caption{Derivada de \(f(z)=3ie^{z}\) en MATLAB}
\end{figure}

% -----------------------------------------------------------------------------
\subsection{Ejercicio 20: Integral de línea}
\subsubsection{Enunciado}
Evalue \(\int_{C}zdz\) donde \(C\) es el segmento de recta de \(0\) a \(1+i\).

\subsubsection{Solución matemática}
Parametrizando el segmento: \(z(t)=t(1+i)\), \(t\in[0,1]\). Entonces:
\[
\int_{C}zdz = \int_{0}^{1} t(1+i) \cdot (1+i)dt = (1+i)^2 \int_{0}^{1} tdt = (2i) \cdot \frac{1}{2} = i
\]

\subsubsection{Resultado}
\[
\int_{C}zdz = i
\]

\subsubsection{Resultado en MATLAB}
\begin{figure}[H]
  \centering
  \includegraphics[width=0.8\textwidth]{figuras/output_N20.png}
  \caption{Integral de línea en MATLAB}
\end{figure}

% -----------------------------------------------------------------------------
\subsection{Ejercicio 21: Curvas de nivel}
\subsubsection{Enunciado}
Grafique las curvas de nivel de Re\((\sin(z))\).

\subsubsection{Solución matemática}
Para \(z = x + iy\), tenemos:
\[
\sin(z) = \sin(x)\cosh(y) + i\cos(x)\sinh(y)
\]
Por lo tanto:
\[
\text{Re}(\sin(z)) = \sin(x)\cosh(y)
\]
Las curvas de nivel son de la forma \(\sin(x)\cosh(y) = c\), donde \(c\) es constante.

\subsubsection{Resultado en MATLAB}
\begin{figure}[H]
  \centering
  \includegraphics[width=0.8\textwidth]{figuras/output_N21.png}
  \caption{Curvas de nivel de Re\((\sin(z))\)}
\end{figure}

% ... Continúa con ejercicios 22-37

\subsection{Ejercicio 22: Cauchy-Riemann para \(z^3\)}
\subsubsection{Enunciado}
Verifique las ecuaciones de Cauchy-Riemann para \(f(z)=z^{3}\).

\subsubsection{Solución matemática}
Sea \(z = x + iy\), entonces:
\[
f(z) = (x + iy)^3 = (x^3 - 3xy^2) + i(3x^2y - y^3)
\]
Identificamos:
\[
u(x,y) = x^3 - 3xy^2, \quad v(x,y) = 3x^2y - y^3
\]
Calculamos las derivadas parciales:
\[
u_x = 3x^2 - 3y^2, \quad u_y = -6xy, \quad v_x = 6xy, \quad v_y = 3x^2 - 3y^2
\]
Se cumple \(u_x = v_y\) y \(u_y = -v_x\), por tanto \(f(z)\) es analítica.

\subsubsection{Resultado en MATLAB}
\begin{figure}[H]
  \centering
  \includegraphics[width=0.9\textwidth]{figuras/output_N22.png}
  \caption{Verificación de Cauchy-Riemann en MATLAB}
\end{figure}

% -----------------------------------------------------------------------------
\subsection{Ejercicio 23: Límite fundamental}
\subsubsection{Enunciado}
Calcule \(\lim_{z\to 0}\)\(\frac{\sin z}{z}\).

\subsubsection{Solución matemática}
Usando el desarrollo en serie de Taylor:
\[
\sin z = z - \frac{z^3}{3!} + \frac{z^5}{5!} - \cdots
\]
Entonces:
\[
\frac{\sin z}{z} = 1 - \frac{z^2}{3!} + \frac{z^4}{5!} - \cdots
\]
Por lo tanto:
\[
\lim_{z\to 0} \frac{\sin z}{z} = 1
\]

\subsubsection{Resultado}
\[
\lim_{z\to 0} \frac{\sin z}{z} = 1
\]

\subsubsection{Resultado en MATLAB}
\begin{figure}[H]
  \centering
  \includegraphics[width=0.8\textwidth]{figuras/output_N23.png}
  \caption{Límite fundamental en MATLAB}
\end{figure}

% -----------------------------------------------------------------------------
\subsection{Ejercicio 24: Mapeo \(w=e^z\)}
\subsubsection{Enunciado}
Grafique el mapeo \(w=e^{z}\) para \(z=x+iy\) con \(x\in[0,1],y\in[0,2\pi]\).

\subsubsection{Solución matemática}
Para \(z = x + iy\), la función exponencial es:
\[
w = e^{z} = e^{x}e^{iy} = e^{x}(\cos y + i\sin y)
\]
El mapeo transforma:
- Líneas verticales (\(x\) constante) en círculos de radio \(e^{x}\)
- Líneas horizontales (\(y\) constante) en rayos con ángulo \(y\)

\subsubsection{Resultado en MATLAB}
\begin{figure}[H]
  \centering
  \includegraphics[width=0.8\textwidth]{figuras/output_N24.png}
  \caption{Mapeo \(w=e^z\) en MATLAB}
\end{figure}

% -----------------------------------------------------------------------------
\subsection{Ejercicio 25: Integral de \(1/z\)}
\subsubsection{Enunciado}
Calcule la integral \(\int_{C}\frac{1}{z}dz\) donde \(C\) es el circulo \(|z|=1\).

\subsubsection{Solución matemática}
Parametrizando el círculo: \(z = e^{i\theta}\), \(\theta \in [0,2\pi]\), \(dz = ie^{i\theta}d\theta\). Entonces:
\[
\int_{C}\frac{1}{z}dz = \int_{0}^{2\pi} \frac{1}{e^{i\theta}} \cdot ie^{i\theta}d\theta = \int_{0}^{2\pi} i d\theta = 2\pi i
\]
Por el teorema de los residuos, el residuo en \(z=0\) es 1, por lo que la integral es \(2\pi i\).

\subsubsection{Resultado}
\[
\int_{C}\frac{1}{z}dz = 2\pi i
\]


% -----------------------------------------------------------------------------
\subsection{Ejercicio 26: Logaritmo complejo}
\subsubsection{Enunciado}
Visualice la parte real e imaginaria de \(f(z)=\log(z)\).

\subsubsection{Solución matemática}
Para \(z = re^{i\theta}\), el logaritmo complejo es:
\[
\log(z) = \ln|z| + i(\theta + 2k\pi), \quad k \in \mathbb{Z}
\]
Para la rama principal (\(k=0\)):
\[
\text{Re}(\log(z)) = \ln r = \frac{1}{2}\ln(x^2 + y^2)
\]
\[
\text{Im}(\log(z)) = \theta = \arctan\left(\frac{y}{x}\right)
\]

\subsubsection{Resultado en MATLAB}
\begin{figure}[H]
  \centering
  \includegraphics[width=0.8\textwidth]{figuras/output_N26.png}
  \caption{Parte real e imaginaria de \(\log(z)\) en MATLAB}
\end{figure}

% -----------------------------------------------------------------------------
\subsection{Ejercicio 27: Derivada de función racional}
\subsubsection{Enunciado}
Calcule la derivada de \(f(z)=\frac{1}{z^{2}+1}\).

\subsubsection{Solución matemática}
Usando la regla del cociente:
\[
f'(z) = \frac{d}{dz}\left(\frac{1}{z^2+1}\right) = -\frac{2z}{(z^2+1)^2}
\]

\subsubsection{Resultado}
\[
f'(z) = -\frac{2z}{(z^2+1)^2}
\]

% -----------------------------------------------------------------------------
\subsection{Ejercicio 28: Integral de \(\bar{z}\)}
\subsubsection{Enunciado}
Evalue \(\int_{C}\bar{z}dz\) donde \(C\) es el circulo \(|z|=2\).

\subsubsection{Solución matemática}
Parametrizando: \(z = 2e^{i\theta}\), \(\theta \in [0,2\pi]\), entonces \(\bar{z} = 2e^{-i\theta}\) y \(dz = 2ie^{i\theta}d\theta\). Luego:
\[
\int_{C}\bar{z}dz = \int_{0}^{2\pi} 2e^{-i\theta} \cdot 2ie^{i\theta}d\theta = \int_{0}^{2\pi} 4i d\theta = 8\pi i
\]

\subsubsection{Resultado}
\[
\int_{C}\bar{z}dz = 8\pi i
\]

% -----------------------------------------------------------------------------
\subsection{Ejercicio 29: Curvas de nivel de \(|1/z|\)}
\subsubsection{Enunciado}
Grafique las curvas de nivel de \(|f(z)|\) para \(f(z)=\frac{1}{z}\).

\subsubsection{Solución matemática}
Para \(z = x + iy\), tenemos:
\[
f(z) = \frac{1}{z} = \frac{x}{x^2+y^2} - i\frac{y}{x^2+y^2}
\]
El módulo es:
\[
|f(z)| = \left|\frac{1}{z}\right| = \frac{1}{|z|} = \frac{1}{\sqrt{x^2+y^2}}
\]
Las curvas de nivel \(|f(z)| = c\) corresponden a círculos \(x^2+y^2 = \frac{1}{c^2}\).

\subsubsection{Resultado en MATLAB}
\begin{figure}[H]
  \centering
  \includegraphics[width=0.8\textwidth]{figuras/output_N29.png}
  \caption{Curvas de nivel de \(|1/z|\) en MATLAB}
\end{figure}

% -----------------------------------------------------------------------------
\subsection{Ejercicio 30: Límite por factorización}
\subsubsection{Enunciado}
Calcule \(\lim_{z\to i}\)\(\frac{z^{2}+1}{z-i}\).

\subsubsection{Solución matemática}
Factorizando el numerador:
\[
z^2 + 1 = (z - i)(z + i)
\]
Entonces:
\[
\frac{z^2+1}{z-i} = \frac{(z-i)(z+i)}{z-i} = z + i
\]
Por lo tanto:
\[
\lim_{z\to i} \frac{z^2+1}{z-i} = i + i = 2i
\]

\subsubsection{Resultado}
\[
\lim_{z\to i} \frac{z^2+1}{z-i} = 2i
\]

% -----------------------------------------------------------------------------
\subsection{Ejercicio 31: Analiticidad de \(\bar{z}\)}
\subsubsection{Enunciado}
Verifique si \(f(z)=\bar{z}\) es analítica.

\subsubsection{Solución matemática}
Sea \(z = x + iy\), entonces:
\[
f(z) = \bar{z} = x - iy
\]
Identificamos:
\[
u(x,y) = x, \quad v(x,y) = -y
\]
Calculamos las derivadas parciales:
\[
u_x = 1, \quad u_y = 0
\]
\[
v_x = 0, \quad v_y = -1
\]
Verificamos Cauchy-Riemann:
\[
u_x = 1 \neq v_y = -1
\]
\[
u_y = 0 = -v_x = 0
\]
Solo se cumple la segunda ecuación. Por tanto, \(f(z)=\bar{z}\) no es analítica.

\subsubsection{Resultado}
\(f(z)=\bar{z}\) no es analítica.

% -----------------------------------------------------------------------------
\subsection{Ejercicio 32: Integral sobre parábola}
\subsubsection{Enunciado}
Calcule \(\int_{C}(z^{2}+1)dz\) donde \(C\) es el arco de parabola \(y=x^{2}\) de \(0\) a \(1+i\).

\subsubsection{Solución matemática}
Parametrizando la parábola: \(z(t) = t + it^2\), \(t \in [0,1]\). Entonces:
\[
dz = (1 + 2it)dt
\]
\[
\int_{C}(z^{2}+1)dz = \int_{0}^{1} [(t + it^2)^2 + 1](1 + 2it)dt
\]
\[
= \int_{0}^{1} [t^2 + 2it^3 - t^4 + 1](1 + 2it)dt
\]
Resolviendo la integral se obtiene el resultado.

\subsubsection{Resultado en MATLAB}
\begin{figure}[H]
  \centering
  \includegraphics[width=0.8\textwidth]{figuras/output_N32.png}
  \caption{Integral sobre parábola en MATLAB}
\end{figure}

% -----------------------------------------------------------------------------
\subsection{Ejercicio 33: Transformación de Joukowski}
\subsubsection{Enunciado}
Grafique la transformacion \(w=z+\frac{1}{z}\).

\subsubsection{Solución matemática}
La transformación \(w = z + \frac{1}{z}\) es conocida como transformación de Joukowski. Para \(z = re^{i\theta}\):
\[
w = re^{i\theta} + \frac{1}{r}e^{-i\theta} = \left(r + \frac{1}{r}\right)\cos\theta + i\left(r - \frac{1}{r}\right)\sin\theta
\]
Transforma círculos en elipses y líneas rectas en hipérbolas.

\subsubsection{Resultado en MATLAB}
\begin{figure}[H]
  \centering
  \includegraphics[width=0.8\textwidth]{figuras/output_N33.png}
  \caption{Transformación de Joukowski en MATLAB}
\end{figure}

% -----------------------------------------------------------------------------
\subsection{Ejercicio 34: Derivada de \(\sin(z^2)\)}
\subsubsection{Enunciado}
Calcule la derivada de \(f(z)=\sin(z^{2})\).

\subsubsection{Solución matemática}
Usando la regla de la cadena:
\[
f'(z) = \frac{d}{dz}[\sin(z^2)] = \cos(z^2) \cdot \frac{d}{dz}(z^2) = 2z\cos(z^2)
\]

\subsubsection{Resultado}
\[
f'(z) = 2z\cos(z^2)
\]

% -----------------------------------------------------------------------------
\subsection{Ejercicio 35: Integral de \(1/z^2\)}
\subsubsection{Enunciado}
Calcule \(\int_{C}\frac{dz}{z^{2}}\) donde \(C\) es el circulo \(|z-1|=1\).

\subsubsection{Solución matemática}
La función \(f(z) = \frac{1}{z^2}\) tiene un polo de orden 2 en \(z=0\). El círculo \(|z-1|=1\) no encierra al polo \(z=0\) (distancia del centro al polo es 1, igual al radio). Como la función es analítica dentro y sobre el contorno:
\[
\int_{C}\frac{dz}{z^{2}} = 0
\]
por el teorema de Cauchy-Goursat.

\subsubsection{Resultado}
\[
\int_{C}\frac{dz}{z^{2}} = 0
\]

% -----------------------------------------------------------------------------
\subsection{Ejercicio 36: Función \(e^{1/z}\)}
\subsubsection{Enunciado}
Grafique la parte real e imaginaria de \(f(z)=e^{1/z}\).

\subsubsection{Solución matemática}
Para \(z = x + iy\), tenemos:
\[
e^{1/z} = e^{\frac{x}{x^2+y^2} - i\frac{y}{x^2+y^2}} = e^{\frac{x}{x^2+y^2}} \left[\cos\left(\frac{y}{x^2+y^2}\right) - i\sin\left(\frac{y}{x^2+y^2}\right)\right]
\]
Por tanto:
\[
\text{Re}(e^{1/z}) = e^{\frac{x}{x^2+y^2}} \cos\left(\frac{y}{x^2+y^2}\right)
\]
\[
\text{Im}(e^{1/z}) = -e^{\frac{x}{x^2+y^2}} \sin\left(\frac{y}{x^2+y^2}\right)
\]

\subsubsection{Resultado en MATLAB}
\begin{figure}[H]
  \centering
  \includegraphics[width=0.8\textwidth]{figuras/output_N36.png}
  \caption{Parte real e imaginaria de \(e^{1/z}\) en MATLAB}
\end{figure}

% -----------------------------------------------------------------------------
\subsection{Ejercicio 37: Límite en infinito}
\subsubsection{Enunciado}
Calcule \(\lim_{z\rightarrow\infty}\)\(\frac{z^{2}+1}{2z^{2}-3z}\).

\subsubsection{Solución matemática}
Dividiendo numerador y denominador por \(z^2\):
\[
\lim_{z\rightarrow\infty} \frac{z^{2}+1}{2z^{2}-3z} = \lim_{z\rightarrow\infty} \frac{1 + \frac{1}{z^2}}{2 - \frac{3}{z}} = \frac{1 + 0}{2 - 0} = \frac{1}{2}
\]

\subsubsection{Resultado}
\[
\lim_{z\rightarrow\infty} \frac{z^{2}+1}{2z^{2}-3z} = \frac{1}{2}
\]

\newpage

% =============================================================================
% ANEXOS
% =============================================================================
\section{Anexos}

\subsection{Anexo A: Códigos MATLAB - Ejercicios 1-18}

\subsubsection{Ejercicio 1 - Operaciones básicas con complejos}
\begin{figure}[H]
  \centering
  \includegraphics[width=0.9\textwidth]{codigos/ejercicio_N1_code.png}
  \caption{Código MATLAB para el Ejercicio 1}
\end{figure}

\subsubsection{Ejercicio 2 - Forma polar de complejo}
\begin{figure}[H]
  \centering
  \includegraphics[width=0.9\textwidth]{codigos/ejercicio_N2_code.png}
  \caption{Código MATLAB para el Ejercicio 2}
\end{figure}

\subsubsection{Ejercicio 3 - Raíces quintas de la unidad}
\begin{figure}[H]
  \centering
  \includegraphics[width=0.9\textwidth]{codigos/ejercicio_N3_code.png}
  \caption{Código MATLAB para el Ejercicio 3}
\end{figure}

\subsubsection{Ejercicio 4 - Ecuaciones de Cauchy-Riemann para $e^z$}
\begin{figure}[H]
  \centering
  \includegraphics[width=0.9\textwidth]{codigos/ejercicio_N4_code.png}
  \caption{Código MATLAB para el Ejercicio 4}
\end{figure}

\subsubsection{Ejercicio 5 - Integral de línea de $z^2$}
\begin{figure}[H]
  \centering
  \includegraphics[width=0.9\textwidth]{codigos/ejercicio_N5_code.png}
  \caption{Código MATLAB para el Ejercicio 5}
\end{figure}

\subsubsection{Ejercicio 6 - Serie de Taylor de $1/(1+z)$}
\begin{figure}[H]
  \centering
  \includegraphics[width=0.9\textwidth]{codigos/ejercicio_N6_code.png}
  \caption{Código MATLAB para el Ejercicio 6}
\end{figure}

\subsubsection{Ejercicio 7 - Mapeo $w=z^2$ para cuadrícula}
\begin{figure}[H]
  \centering
  \includegraphics[width=0.9\textwidth]{codigos/ejercicio_N7_code.png}
  \caption{Código MATLAB para el Ejercicio 7}
\end{figure}

\subsubsection{Ejercicio 8 - Límite complejo $((n+1)/n)^{ni}$}
\begin{figure}[H]
  \centering
  \includegraphics[width=0.9\textwidth]{codigos/ejercicio_N8_code.png}
  \caption{Código MATLAB para el Ejercicio 8}
\end{figure}

\subsubsection{Ejercicio 9 - Curvas de nivel de $\text{Re}(e^{1/z})$}
\begin{figure}[H]
  \centering
  \includegraphics[width=0.9\textwidth]{codigos/ejercicio_N9_code.png}
  \caption{Código MATLAB para el Ejercicio 9}
\end{figure}

\subsubsection{Ejercicio 10 - Raíces de $z^4 = 16i$}
\begin{figure}[H]
  \centering
  \includegraphics[width=0.9\textwidth]{codigos/ejercicio_N10_code.png}
  \caption{Código MATLAB para el Ejercicio 10}
\end{figure}

\subsubsection{Ejercicio 11 - Integral de $e^{i\theta}$}
\begin{figure}[H]
  \centering
  \includegraphics[width=0.9\textwidth]{codigos/ejercicio_N11_code.png}
  \caption{Código MATLAB para el Ejercicio 11}
\end{figure}

\subsubsection{Ejercicio 12 - Integral de $\cos(z)/z$ por residuos}
\begin{figure}[H]
  \centering
  \includegraphics[width=0.9\textwidth]{codigos/ejercicio_N12_code.png}
  \caption{Código MATLAB para el Ejercicio 12}
\end{figure}

\subsubsection{Ejercicio 13 - Derivada de $\log(z)$}
\begin{figure}[H]
  \centering
  \includegraphics[width=0.9\textwidth]{codigos/ejercicio_N13_code.png}
  \caption{Código MATLAB para el Ejercicio 13}
\end{figure}

\subsubsection{Ejercicio 14 - Serie de Taylor de $e^z$}
\begin{figure}[H]
  \centering
  \includegraphics[width=0.9\textwidth]{codigos/ejercicio_N14_code.png}
  \caption{Código MATLAB para el Ejercicio 14}
\end{figure}

\subsubsection{Ejercicio 15 - Raíces de $z^3 + z^2 + z + 1 = 0$}
\begin{figure}[H]
  \centering
  \includegraphics[width=0.9\textwidth]{codigos/ejercicio_N15_code.png}
  \caption{Código MATLAB para el Ejercicio 15}
\end{figure}

\subsubsection{Ejercicio 16 - Mapeo $w=\sin(z)$}
\begin{figure}[H]
  \centering
  \includegraphics[width=0.9\textwidth]{codigos/ejercicio_N16_code.png}
  \caption{Código MATLAB para el Ejercicio 16}
\end{figure}

\subsubsection{Ejercicio 17 - Integral de $1/(z^2+4)$ por residuos}
\begin{figure}[H]
  \centering
  \includegraphics[width=0.9\textwidth]{codigos/ejercicio_N17_code.png}
  \caption{Código MATLAB para el Ejercicio 17}
\end{figure}

\subsubsection{Ejercicio 18 - Gráfica de $f(z)=z^2$}
\begin{figure}[H]
  \centering
  \includegraphics[width=0.9\textwidth]{codigos/ejercicio_N18_code.png}
  \caption{Código MATLAB para el Ejercicio 18}
\end{figure}

% =============================================================================

\newpage
\subsection{Anexo B: Códigos MATLAB - Ejercicios 19-37}

\subsubsection{Ejercicio 19 - Derivada de $f(z)=3ie^{z}$}
\begin{figure}[H]
  \centering
  \includegraphics[width=0.9\textwidth]{codigos/ejercicio_N19_code.png}
  \caption{Código MATLAB para el Ejercicio 19}
\end{figure}

\subsubsection{Ejercicio 20 - Integral de línea de $z$ sobre segmento}
\begin{figure}[H]
  \centering
  \includegraphics[width=0.9\textwidth]{codigos/ejercicio_N20_code.png}
  \caption{Código MATLAB para el Ejercicio 20}
\end{figure}

\subsubsection{Ejercicio 21 - Curvas de nivel de $\text{Re}(\sin(z))$}
\begin{figure}[H]
  \centering
  \includegraphics[width=0.9\textwidth]{codigos/ejercicio_N21_code.png}
  \caption{Código MATLAB para el Ejercicio 21}
\end{figure}

\subsubsection{Ejercicio 22 - Cauchy-Riemann para $f(z)=z^3$}
\begin{figure}[H]
  \centering
  \includegraphics[width=0.9\textwidth]{codigos/ejercicio_N22_code.png}
  \caption{Código MATLAB para el Ejercicio 22}
\end{figure}

\subsubsection{Ejercicio 23 - Límite fundamental $\lim_{z\to 0} \frac{\sin z}{z}$}
\begin{figure}[H]
  \centering
  \includegraphics[width=0.9\textwidth]{codigos/ejercicio_N23_code.png}
  \caption{Código MATLAB para el Ejercicio 23}
\end{figure}

\subsubsection{Ejercicio 24 - Mapeo $w=e^z$}
\begin{figure}[H]
  \centering
  \includegraphics[width=0.9\textwidth]{codigos/ejercicio_N24_code.png}
  \caption{Código MATLAB para el Ejercicio 24}
\end{figure}

\subsubsection{Ejercicio 25 - Integral de $1/z$ alrededor de $|z|=1$}
\begin{figure}[H]
  \centering
  \includegraphics[width=0.9\textwidth]{codigos/ejercicio_N25_code.png}
  \caption{Código MATLAB para el Ejercicio 25}
\end{figure}

\subsubsection{Ejercicio 26 - Parte real e imaginaria de $\log(z)$}
\begin{figure}[H]
  \centering
  \includegraphics[width=0.9\textwidth]{codigos/ejercicio_N26_code.png}
  \caption{Código MATLAB para el Ejercicio 26}
\end{figure}

\subsubsection{Ejercicio 27 - Derivada de $f(z)=\frac{1}{z^2+1}$}
\begin{figure}[H]
  \centering
  \includegraphics[width=0.9\textwidth]{codigos/ejercicio_N27_code.png}
  \caption{Código MATLAB para el Ejercicio 27}
\end{figure}

\subsubsection{Ejercicio 28 - Integral de $\bar{z}$ alrededor de $|z|=2$}
\begin{figure}[H]
  \centering
  \includegraphics[width=0.9\textwidth]{codigos/ejercicio_N28_code.png}
  \caption{Código MATLAB para el Ejercicio 28}
\end{figure}

\subsubsection{Ejercicio 29 - Curvas de nivel de $|1/z|$}
\begin{figure}[H]
  \centering
  \includegraphics[width=0.9\textwidth]{codigos/ejercicio_N29_code.png}
  \caption{Código MATLAB para el Ejercicio 29}
\end{figure}

\subsubsection{Ejercicio 30 - Límite por factorización $\lim_{z\to i} \frac{z^2+1}{z-i}$}
\begin{figure}[H]
  \centering
  \includegraphics[width=0.9\textwidth]{codigos/ejercicio_N30_code.png}
  \caption{Código MATLAB para el Ejercicio 30}
\end{figure}

\subsubsection{Ejercicio 31 - Analiticidad de $f(z)=\bar{z}$}
\begin{figure}[H]
  \centering
  \includegraphics[width=0.9\textwidth]{codigos/ejercicio_N31_code.png}
  \caption{Código MATLAB para el Ejercicio 31}
\end{figure}

\subsubsection{Ejercicio 32 - Integral sobre parábola $y=x^2$}
\begin{figure}[H]
  \centering
  \includegraphics[width=0.9\textwidth]{codigos/ejercicio_N32_code.png}
  \caption{Código MATLAB para el Ejercicio 32}
\end{figure}

\subsubsection{Ejercicio 33 - Transformación de Joukowski $w=z+1/z$}
\begin{figure}[H]
  \centering
  \includegraphics[width=0.9\textwidth]{codigos/ejercicio_N33_code.png}
  \caption{Código MATLAB para el Ejercicio 33}
\end{figure}

\subsubsection{Ejercicio 34 - Derivada de $f(z)=\sin(z^2)$}
\begin{figure}[H]
  \centering
  \includegraphics[width=0.9\textwidth]{codigos/ejercicio_N34_code.png}
  \caption{Código MATLAB para el Ejercicio 34}
\end{figure}

\subsubsection{Ejercicio 35 - Integral de $1/z^2$ alrededor de $|z-1|=1$}
\begin{figure}[H]
  \centering
  \includegraphics[width=0.9\textwidth]{codigos/ejercicio_N35_code.png}
  \caption{Código MATLAB para el Ejercicio 35}
\end{figure}

\subsubsection{Ejercicio 36 - Parte real e imaginaria de $e^{1/z}$}
\begin{figure}[H]
  \centering
  \includegraphics[width=0.9\textwidth]{codigos/ejercicio_N36_code.png}
  \caption{Código MATLAB para el Ejercicio 36}
\end{figure}

\subsubsection{Ejercicio 37 - Límite en infinito $\lim_{z\to\infty} \frac{z^2+1}{2z^2-3z}$}
\begin{figure}[H]
  \centering
  \includegraphics[width=0.9\textwidth]{codigos/ejercicio_N37_code.png}
  \caption{Código MATLAB para el Ejercicio 37}
\end{figure}

% =============================================================================

\subsection{Notas sobre los códigos}

\begin{itemize}
\item Todos los códigos fueron desarrollados en MATLAB R2023a.
\item Se utilizó el \texttt{Symbolic Math Toolbox} para los cálculos simbólicos.
\item Los gráficos fueron generados con las funciones de visualización estándar de MATLAB.
\item Las capturas muestran el código completo utilizado para cada ejercicio.
\item Los códigos están optimizados para claridad y eficiencia computacional.
\item La mayoria de codigos fueron proporcionados por repositorios.
\end{itemize}

\end{document}